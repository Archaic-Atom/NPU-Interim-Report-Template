%% File: npu-report-sample.tex (西北工业大学研究生学位论文选题报告表)
%% Date: 2018/01/20
%% Author: Yang Zongze  (yangzongze@gmail.com)
\documentclass{article}
\usepackage{npu-report-style}
\usepackage[pagebackref=false, colorlinks, linkcolor=red, anchorcolor=yellow, citecolor=blue, bookmarks=false ]{hyperref}
\usepackage{graphicx, subfig, epstopdf, amsmath, booktabs, paralist, float, caption, xfrac, enumitem, array, multirow, cite}
\usepackage{gbt7714}
\usepackage{zhnumber}
\usepackage{titlesec}


\newcommand{\erhao}{\fontsize{12pt}{\baselineskip}\selectfont}
\titleformat{\section}{\erhao\bf}{}{0em}{}[]
\renewcommand\figureautorefname{图}		% 重新定义引用图标
\renewcommand\tableautorefname{表}		% 重新定义应用表格
\def\equationautorefname{式}				% 重新定义公式

%% 设置 `论文题目' 
\nputitle{遥感图像的语义分割与场景重建研究}
\npunumber{2017100580}
\npuschool{电子信息学院}
\npumajor{信息与通信工程}
\npuname{饶智博}
\npudegree{博士}
\npusupervisor{何明一\ 教授}
\nputype{全日制学术型}
\npusubjectsource{国家自然科学基金}
\npudate{\today}
%% 设置 `论文类型': 取消或添加注释即可勾选相应类型
%\npucheckbasetrue   	%  基础研究
\npucheckapplytrue     	%  应用研究
\npucheckenginetrue   	%  工程技术
%\npucheckoveralltrue   	%  跨学科研究
\addachievement{1}{Bidirectional guided attention network for 3-d semantic detection of remote sensing images (已发表, SCI 1 区,学院标准 A类)}{SCI期刊论文(TGRS)}{1}
\addachievement{2}{Patch attention network with generative adversarial model for semi-supervised binocular disparity prediction (已发表,  SCI 4 区)}{SCI期刊论文(TJVC)}{1}
\addachievement{3}{Nlca-net: a non-local context attention network for stereo matching (已发表)}{EI期刊论文(ATSIP)}{1}
\addachievement{4}{Msdc-net: Multi-scale dense and contextual networks for stereo matching (已发表)}{EI会议(APSIPA ASC)}{1}
\addachievement{5}{Sdbf-net: Semantic and disparity bidirectional fusion network for 3d semantic detection on incidental satellite images(已发表)}{EI会议(APSIPA ASC)}{1}
\addachievement{6}{Input-perturbation-sensitivity for performance analysis of cnns on image recognition}{EI会议(ICIP)}{1}
\addachievement{7}{Class Attention Network for Semantic Segmentation of Remote Sensing Images(已发表)}{EI会议(APSIPA ASC)}{1}
\addachievement{8}{Image2Mesh: 3D mesh reconstruction framework from pairwise remote sensing images (投稿中)}{EI会议}{1}
\addachievement{9}{Rethinking Pre-training and Data Augmentation in Robust Vision Challenge of Stereo Matching(投稿中)}{SCI期刊}{1}
\addachievement{10}{ECCV Robust Challenge 2020 (stereo matching 赛道)}{国际竞赛第二名}{1}
\addachievement{11}{CFNet: Cascade and Fused Cost Volume for Efficient and Robust Stereo Matching (已发表)}{EI会议(CVPR)}{3}
\addachievement{12}{Mvs2: Deep unsupervised multi-view stereo with multi-view symmetry(已发表)}{EI会议(3DV)}{3}
\addachievement{13}{Multi-scale cross-form pyramid network for stereo matching (已发表)}{EI会议(ICIEA)}{4}
\addachievement{14}{TPMT based Automatic Road Extraction from 3D Real Scenes (已发表)}{EI会议(BGDDS)}{4}
\addachievement{15}{一种基于PC/104嵌入式系统的旋翼共锥度机载测量装置及方法 (已授权)}{发明专利}{3}
\addachievement{16}{MSMD-Net: Deep Stereo Matching with Multi-scale and Multi-dimension Cost Volume (投稿中)}{SCI期刊}{3}
\addachievement{17}{Sar and optical image template matching based on neural network (投稿中)}{EI期刊(会议)}{3}
\npuachievementnum{17}

\begin{document}
\section*{一、哪些研究内容已按时或提前完成,主要进展和成果。}
\begin{enumerate}
\item 针对遥感图像中标签分布不均衡的问题,本研究提出了类注意力网络(Class attention network, CA-Net)。该方法利用注意力机制关注遥感图像中小型目标物体,并利用多种数据增强和Focal损失函数较好地缓解了遥感图像中标签不平衡的问题。本研究提出的语义分割算法有效地提高了算法的分类精度,尤其是小目标物体的分类结果。\textbf{该研究成果已经发表在EI会议中}\cite{Rao2020Class}(2020 Asia-Pacific Signal and Information Processing Association Annual Summit and Conference, APSIPA ASC)。
\item 针对传统立体匹配算法中反光区域匹配难和遮挡区域误匹配问题,本研究首先提出多尺度稠密上下文网络(Multi-scale dense and contextual networks, MSDC-Net)。该方法利用多尺度特征构建匹配代价量(Cost volume),有效地增强了网络的场景能力,从而提高了特定区域的匹配精度。\textbf{该研究成果已经发表在EI会议中}\cite{Rao2019Msdc}(2019 Asia-Pacific Signal and Information Processing Association Annual Summit and Conference, APSIPA ASC)。在该项目基础上,本研究引入非局部注意力机制(Non-local attention mechanism),从而进一步提升了匹配的精度和难点区域的表现。\textbf{该研究成果已经发表在EI期刊中}\cite{Rao2020Nlca}(APSIPA Transactions on Signal and Information Processing, ATSIP),\textbf{并获得了ECCV鲁棒视觉挑战赛(ECCV Robust Challenge 2020)的第二名}。
\item 针对遥感图像立体匹配算法中四季变换问题,本研究首先研究了语义和视差的融合方法,称之为语义视差双向融合网络(Semantic and disparity bidirectional fusion network, SDBF-net )。该方法证明语义和视差信息能够互相促进。\textbf{该研究成果已经发表在EI会议中}\cite{Rao2019Sdbf}(2019 Asia-Pacific Signal and Information Processing Association Annual Summit and Conference, APSIPA ASC)。在此项目的基础上,本研究提出了一个语义视差的多任务学习框架,该框架成为双向注意力引导网络(Bidirectional Guided Attention Network, BGA-Net)。该方法能同时进行语义分类和视差估计,并证明语义信息能够有效地提高网络的场景理解能力,从而克服遥感场景下四季变换问题,进而提高遥感图像的匹配精度。\textbf{该方法取得了US3D排行榜的第一,且相关成果已经发表在SCI期刊中}\cite{Rao2020BiDirectional}(IEEE Transactions on Geoscience and Remote Sensing, TGRS)。
\item 在现有研究基础上,针对现有的深度学习模型非常依赖大数据训练的问题,本研究提出了一个适用于匹配问题的半监督模型。该方法利用生成对抗网络生成样本和鉴别样本的能力,构建了一个半监督网络模型。其次,采用一种块注意力方式代替之前常用的三维卷积,以减少模型参数和使用内存。\textbf{该研究成果已经发表在SCI期刊中}\cite{Rao2020Patch}(The Visual Computer)。
\end{enumerate}

\section{二、哪些研究内容未按计划完成,原因何在?}
无

\section{三、存在的问题和需要说明的情况(调整变动内容)。}
无

\section{四、后续研究工作安排。}
\begin{enumerate}
\item 对正在审稿的论文,按要求进行规整、补充和完善,力争将正在投稿中的论文转变成已发表论文;
\item 将已发表的论文进行规整,并完成大论文的撰写与修改;
\item 研究项目总结验收,论文修改及答辩。
\end{enumerate}

\bibliographystyle{gbt7714-numerical}
\bibliography{Cite}
\end{document}


